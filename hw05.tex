\documentclass[12pt,a4paper]{scrartcl}
\usepackage[utf8]{inputenc}
\usepackage[english,russian]{babel}
\usepackage{indentfirst}
\usepackage{misccorr}
\usepackage{graphicx}
\usepackage{amsmath}
\usepackage{ marvosym }
\usepackage{color,soul}
\usepackage{upgreek}
\usepackage{gensymb}

% Основные математические символы
\DeclareSymbolFont{extraup}{U}{zavm}{m}{n}
\DeclareMathSymbol{\heart}{\mathalpha}{extraup}{86}
\newcommand{\N}{\mathbb{N}}   % Natural numbers
\newcommand{\R}{\mathbb{R}}   % Ratio numbers
\newcommand{\Z}{\mathbb{Z}}   % Integer numbers
\def\INF{\t{+}\infty}         % +inf
\def\EPS{\varepsilon}         %
\def\EMPTY{\varnothing}       %
\def\PHI{\varphi}             %
\def\SO{\Rightarrow}          % =>
\def\EQ{\Leftrightarrow}      % <=>
\def\t{\texttt}               % mono font
\def\c#1{{\rm\sc{#1}}}        % font for classes NP, SAT, etc
\def\O{\mathcal{O}}           %
\def\NO{\t{\#}}               % #
\renewcommand{\le}{\leqslant} % <=, beauty
\renewcommand{\ge}{\geqslant} % >=, beauty
\def\XOR{\text{ {\raisebox{-2pt}{\ensuremath{\Hat{}}}} }}
\newcommand{\q}[1]{\langle #1 \rangle}               % <x>
\newcommand\URL[1]{{\footnotesize{\url{#1}}}}        %
\newcommand{\sfrac}[2]{{\scriptstyle\frac{#1}{#2}}}  % Очень маленькая дробь
\newcommand{\mfrac}[2]{{\textstyle\frac{#1}{#2}}}    % Небольшая дробь
\newcommand{\score}[1]{{\bf\color{red}{(#1)}}}
\newcommand{\hilight}[1]{\colorbox{yellow}{#1}}

% Отступы
\def\makeparindent{\hspace*{\parindent}}
\def\up{\vspace*{-0.3em}}
\def\down{\vspace*{0.3em}}
\def\LINE{\vspace*{-1em}\noindent \underline{\hbox to 1\textwidth{{ } \hfil{ } \hfil{ } }}}
%\def\up{\vspace*{-\baselineskip}}

\newenvironment{smallformula}{
	
	\vspace*{-0.8em}
}{
	\vspace*{-1.2em}
	
}
\newenvironment{formula}{
	
	\vspace*{-0.4em}
}{
	\vspace*{-0.6em}
	
}

\begin{document}
	\section*{Формальные языки}
	\subsection*{Домашнее задание 5}
	\begin{flushright}
		Фадеева Екатерина
	\end{flushright}

\begin{description}
	\item[\fbox{2.}]
	$\begin{cases}
	\text{S} \rightarrow \text{R S | R}\\
	\text{R} \rightarrow \text{a S b | c R d | a b | c d | } \epsilon
	\end{cases}$
	
	Уберём длинные правила:
	
	$\begin{cases}
	\text{S} \rightarrow \text{R S | R}\\
	\text{R} \rightarrow \text{a X | c Y | a b | c d | } \epsilon\\
	\text{X} \rightarrow \text{S b}\\
	\text{Y} \rightarrow \text{R d}
	\end{cases}$
	
	Удалим $\epsilon$-правила:
	
	$\begin{cases}
	\text{S} \rightarrow \text{R S | R | S | } \epsilon\\
	\text{R} \rightarrow \text{a X | c Y | a b | c d}\\
	\text{X} \rightarrow \text{S b}\\
	\text{Y} \rightarrow \text{R d | d}
	\end{cases}$
	
	Создадим новое стартовое состояние:
	
	$\begin{cases}
	\text{S} \rightarrow \text{S' | } \epsilon\\
	\text{S'} \rightarrow \text{R S' | R | S' } \\
	\text{R} \rightarrow \text{a X | c Y | a b | c d}\\
	\text{X} \rightarrow \text{S' b}\\
	\text{Y} \rightarrow \text{R d | d}
	\end{cases}$
	
	
	Удалим цепные правила (цепные пары --- $(S, S')$, $(S', S')$, $(S', R)$):
	
	$\begin{cases}
	\text{S} \rightarrow \text{R S' | a X | c Y | a b | c d | } \epsilon\\
	\text{S'} \rightarrow \text{R S' | a X | c Y | a b | c d } \\
	\text{R} \rightarrow \text{a X | c Y | a b | c d}\\
	\text{X} \rightarrow \text{S' b}\\
	\text{Y} \rightarrow \text{R d | d}
	\end{cases}$	
	
	Удалять бесполезные нетерминалы не нужно: все символы достижимы (например из $S$ за один шаг), все нетерминалы порождающие ($\text{S} \rightarrow \text{ab}, \text{S'} \rightarrow \text{ab}, \text{R} \rightarrow \text{ab}, \text{X} \rightarrow \text{S'b} \rightarrow \text{abb}, \text{Y} \rightarrow \text{d}$).
	
	Уберем правила из нетерминала в терминалы:
	
	$\begin{cases}
	\text{S} \rightarrow \text{R S' | A X | C Y | A B | C D | } \epsilon\\
	\text{S'} \rightarrow \text{R S' | A X | C Y | A B | C D } \\
	\text{R} \rightarrow \text{A X | C Y | A B | C D}\\
	\text{X} \rightarrow \text{S' B}\\
	\text{Y} \rightarrow \text{R D | D}\\
	\text{A} \rightarrow \text{a}\\
	\text{B} \rightarrow \text{b}\\
	\text{C} \rightarrow \text{c}\\
	\text{D} \rightarrow \text{d}
	\end{cases}$ \\--- грамматика в нормальной форме Хомского.


	\item[\fbox{3.}] Язык $\{ a^m b^n\,\,|\,\,m + n > 0, m + n\,\,\vdots\,\, 2 \}$ явяется контекстно-свободным, грамматика:
	
	$\begin{cases}
		\text{S} \rightarrow \text{a a S | S b b | a S b | a b | a a | b b}
	\end{cases}$
	
	Любые строки, которые описывает эта грамматика, --- одна из строк $\text{a b | a a | b b}$, к которой сколько угодно раз проделывали какие-то из этих операций:
	
	\begin{enumerate}
		\item дописывали слева две буквы $a$
		
		\item дописывали справа две буквы $b$
	
		\item дописывали слева $a$ и справа $b$.
	\end{enumerate}
	
	Тогда в любой такой строке после букв $a$ следуют буквы $b$, и их суммарное количество четно (т.к. изначатьно во всех строках $\text{a b | a a | b b}$ оно четно и мы дописываем только четное количество символов).
	
	В другую сторону: для любой строки $a^mb^n$ при $m+n > 0$ и $\vdots \,\, 2$:
	
	\begin{enumerate}
		\item либо в ней $n=0$, а значит $m \,\,\vdots\,\,2$, т.е. эту строку можно сделать из $aa$ несколькими добавлениями слева букв $aa$
		\item либо в ней $m=0$, тогда можно сделать аналогично добавлениями к $bb$ букв $bb$ справа
		\item либо $n \neq 0, m \neq 0$, тогда можно сделать такую строку из строки $ab$ добавлениями букв $aa$ слева ($\lfloor\frac{m-1}{2}\rfloor$ раз) и $bb$ справа ($\lfloor\frac{n-1}{2}\rfloor$ раз) и если $n \vdots 2$ и $m \vdots 2$, то еще одним добавлением $a$ слева и $b$ справа.
	\end{enumerate}

	Т.о. приведенная грамматика содержит все строки исходного языка и не содержит других строк, значит она описывает этот язык.
	
	
\end{description}

\end{document}